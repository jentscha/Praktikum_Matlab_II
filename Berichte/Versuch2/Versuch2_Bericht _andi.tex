% =================================================================================
% Hier ausw�hlen, ob TUD-Design oder nicht
% =================================================================================
\newif\ifTUDdesign
\TUDdesigntrue					% TUD-Design
%\TUDdesignfalse				% F�r Rechner ohne installierte TUDdesign-Pakete
% =================================================================================


% =================================================================================
% Hier Daten f�r studentische Arbeit eingeben
% =================================================================================
\newcommand{\SADATyp}{Praktikumsbericht}
\newcommand{\SADATitel}{Versuch II: Linearisierung, Steuerbarkeit und Beobachtbarkeit}
\newcommand{\SADAStadt}{Darmstadt}
\newcommand{\SADAAutor}{Andreas Jentsch, Ali Kerem Sacakli}
\newcommand{\SADABetreuer}{}
\newcommand{\SADABetreuerII}{}
\newcommand{\SADABetreuerIII}{}
\newcommand{\SADABegin}{}
\newcommand{\SADAAbgabe}{Praktikum Matlab/Simulink II}
\newcommand{\SADASeminar}{}
% =================================================================================


% =================================================================================
% Auswahl des IAT-Fachgebiets (rtm / rtp)
% =================================================================================
\newif\ifrtm
\rtmtrue	% rtm
%\rtmfalse	% rtp
% =================================================================================


% =================================================================================
% Erkl�rung, dass die Arbeit ohne Hilfe Dritter etc. erstellt wurde
% =================================================================================
\def\SADAVarianteErklaerung{ETIT}		% FB 18, Elektrotechnik
%\def\SADAVarianteErklaerung{MBDA}		% FB 16, Maschinenbau, Diplomarbeit
%\def\SADAVarianteErklaerung{MBSA}		% FB 16, Maschinenbau, Studienarbeit
% =================================================================================


% =================================================================================
% Ausnahmen von der automatischen Silbentrennung
% =================================================================================
\hyphenation{Aktu-ali-sie-rung Screen-shots Pa-rallel-ro-bo-ter Zu-stands-raum-mo-del-le nach-voll-zieh-bar Pro-jekt-se-mi-nar}
% =================================================================================


% =================================================================================
% Hier NIICHTS �ndern!
% =================================================================================
\ifTUDdesign
	\documentclass[11pt, twoside, colorback, accentcolor=tud2c, nopartpage, bigchapter, fleqn, ngerman, longdoc]{tudreport}
\else
	\documentclass[11pt, a4paper, twoside, fleqn, ngerman]{scrreprt}
  % F�r Entwurf auf Rechnern ohne installierte TUDdesign-Pakete	
	\usepackage{exscale}	% Korrektur math-Zeichen
	\usepackage{eurosym}
\fi
%eps-plot von Matlab in pdf umwandeln, um diese einbinden zu k�nnen
\usepackage{epstopdf}
%Paket listings f�r das Einbinden von Quelltext
\usepackage{listings}
%Stil Matlab_colored definieren
\lstdefinestyle{Matlab_colored}
{
	language = Matlab,
	tabsize = 4,
	framesep = 3mm,
	frame = tb,
	classoffset = 0,
	basicstyle = \ttfamily,
	keywordstyle = \bfseries\color[rgb]{0,0,1},
	commentstyle = \itshape\color[rgb]{0.133,0.545,0.133},
	stringstyle = \color[rgb]{0.672,0.126,0.941},
	extendedchars = true,
	breaklines = true,
	prebreak = \textrightarrow,
	postbreak = \textleftarrow,
	numbers = left,
	numberstyle = \tiny,
	stepnumber = 5
}


\input{common/includes.tex}				% verwendete Pakete einbinden
\input{common/setup.tex}					% LaTeX-Einstellungen
\input{common/commonmacros.tex}		% oft verwendete Befehle
% =================================================================================


% =================================================================================
% Hier beginnt das eigentliche Dokument
% =================================================================================

\begin{document}
%\input{common/preface.tex} % Titelseite, Aufgabenstellung, Erkl�rung, Abstract, Inhaltsverzeichnis, etc.
\maketitle

\setcounter{chapter}{2}

%Warum geht title/section nicht?
\section{Lienarisierung}
Im folgenden Abschnitt wird die Funktion zur Linearisierung des Doppelpendel-Systems um einen Arbeitspunkt, sowie ihre R�ckgabewerte an bestimmten Arbeitspunkten dokumentiert. Die Implementierung der Funktion ist in \autoref{Code_Linearisierung} aufgef�hrt.

Bevor das System linearisiert wird sind zwei Fragen zu kl�ren:
\begin{enumerate}
	\item Welche Arbeitspunkte sind sinnvoll?
	\item Was bedeutet es physikalisch, wenn $M_{\textrm{AP}}$ ungleich null ist?
\end{enumerate}
Die Antworten lauten wie folgt:
\begin{enumerate}
	\item Es ist nur sinnvoll das System in Arbeitspunkten zu linearisieren, in denen es sowohl vollst�ndig beobachtbar, als auch steuerbar ist.
	\item Bei der Gr��e $M_{\textrm{AP}}$ handelt es sich um den statischen Wert der Stellgr��e $M$ im Arbeitspunkt. Ist diese ungleich null muss der Motor das Moment $M_{\textrm{}}$ 
\end{enumerate}

\lstinputlisting[style=Matlab_colored,caption={Code der Linearisierungsfunktion}, label={Code_Linearisierung}]{Codes/linearisierung.m}

Die Linearisierung um die Arbeitspunkte
\begin{align*}
	\textbf{x}_{\textrm{AP}_1} &= \begin{bmatrix} 0 & 0 & 0 & 0\end{bmatrix}\\
	\textbf{x}_{\textrm{AP}_2} &= \begin{bmatrix} \pi & 0 & \pi & 0\end{bmatrix}\\
	\textbf{x}_{\textrm{AP}_3} &= \begin{bmatrix} \pi/2 & 0 & \pi & 0 \end{bmatrix}
\end{align*}
ergibt f�r die allgmeine Zustandsraumdarstellung:
\begin{align*}
	\textbf{\.{x}} &= \textbf{Ax + Bu}\\
	\textbf{y} &= \textbf{Cx + Du}
\end{align*}
die folgenden Systemmatrizen:
\begin{align*}
\textbf{A}_{\textrm{AP}1} &= \begin{bmatrix} 0& 1 & 0& 0\\ -126,1286& 0 & 63,0643 & 0\\ 0&0&0&1\\189,1929&0&168,1714&0 \end{bmatrix}&
\textbf{B}_{\textrm{AP}1} &= \begin{bmatrix} 0\\142,8571\\0\\-214,2857\end{bmatrix}\\
\textbf{C}_{\textrm{AP}1} &= \begin{bmatrix} 1&0&0&0\\0&0&1&0\end{bmatrix}&
\textbf{D}_{\textrm{AP}1} &= \begin{bmatrix} 0\\0\end{bmatrix}\\
\\
\textbf{A}_{\textrm{AP}2} &= \begin{bmatrix} 0& 1 & 0& 0\\ 126,1286& 0 & -63,0643 & 0\\ 0&0&0&1\\-189,1929&0&168,1714&0 \end{bmatrix}&
\textbf{B}_{\textrm{AP}2} &= \begin{bmatrix} 0\\142,8571\\0\\-214,2857\end{bmatrix}\\
\textbf{C}_{\textrm{AP}2} &= \begin{bmatrix} 1&0&0&0\\0&0&1&0\end{bmatrix}&
\textbf{D}_{\textrm{AP}2} &= \begin{bmatrix} 0\\0\end{bmatrix}\\
\\
\textbf{A}_{\textrm{AP}3} &= \begin{bmatrix} 0& 1 & 0& 0\\ 0& 0 & 0 & 0\\ 0&0&0&1\\0&0&73,5750&0 \end{bmatrix}&
\textbf{B}_{\textrm{AP}3} &= \begin{bmatrix} 0\\62,5\\0\\0\end{bmatrix}\\
\textbf{C}_{\textrm{AP}3} &= \begin{bmatrix} 1&0&0&0\\0&0&1&0\end{bmatrix}&
\textbf{D}_{\textrm{AP}3} &= \begin{bmatrix} 0\\0\end{bmatrix}
\end{align*}
\section{Vergleich der Linearisierten Modelle}
Die Eigenwerte der Zustandsraummodelle um die drei Arbeitspunkte lauten wie folgt:
\begin{align*}
	\mathbf{\lambda}_1 &= \begin{bmatrix}16,0744i\\-16,0744i\\5,9929i\\-5,9929i\end{bmatrix}\\
	\mathbf{\lambda}_2 &= \begin{bmatrix}-16,0744\\16,0744\\-5,9929\\5,9929\end{bmatrix}\\
	\mathbf{\lambda}_3 &= \begin{bmatrix}8,5776\\-8,5776\\0\\0\end{bmatrix}
\end{align*}

System 1 ist unged�mpft schwingungsf�hig und grenzstabil. System 2 ist instabil. System 3 ist instabil.

\section{Normalformen des Zustandsraummodelles}
\lstinputlisting[style=Matlab_colored,caption={Code der Diagonalisierungsfunktion}, label={Code_diagonalForm}]{Codes/diagonalForm.m}

Die Transformation des um AP2 linearisierten Systems in die Diagonalform mit der Funktion in \autoref{Code_diagonalForm} ergibt folgende Systemmatritzen:
\begin{align*}
\textbf{A}_{\textrm{D}} &= \begin{bmatrix} -16,0744& 0 & 0& 0\\ 0& 16,0744 & 0 & 0\\ 0&0&-5,9929&0\\0&0&&5,9929\end{bmatrix}&
\textbf{B}_{\textrm{D}} &= \begin{bmatrix} 138,1293\\138,1293\\-21,3960\\21,3960\end{bmatrix}\\
\textbf{C}_{\textrm{D}} &= \begin{bmatrix} -0,0267&0,0267&0,0943&0,0943\\0,0560&-0,0560&0,1349&0,1349\end{bmatrix}&
\mathbf{D}_{\mathrm{D}} &= \begin{bmatrix} 0\\0\end{bmatrix}
\end{align*}

Die Transormation des selben Systems mit der MATLAB Funktion \lstinline[style=Matlab_colored]|canon| ergibt folgende Systemmmatritzen:
\begin{align*}
\textbf{A}_{\textrm{D}} &= \begin{bmatrix} -16,0744& 0 & 0& 0\\ 0& -5,9929 & 0 & 0\\ 0&0&16,0744&0\\0&0&&5,9929\end{bmatrix}&
\textbf{B}_{\textrm{D}} &= \begin{bmatrix} 2,5354\\-0,7024\\2,5354\\0,7024\end{bmatrix}\\
\textbf{C}_{\textrm{D}} &= \begin{bmatrix} -1,4559&2,8726&1,4559&2,8726\\3,0534&4,1093&-3,0534&4,1093\end{bmatrix}&
\mathbf{D}_{\mathrm{D}} &= \begin{bmatrix} 0\\0\end{bmatrix}
\end{align*}

Die beiden Transformierten Systeme unterscheiden sich in der Sortierung der Eigenwerte in der $\mathbf{A}$ matrix, sowie in den Werten der $\mathbf{B}$ und $\mathbf{C}$ Matritzen. Dies ist durch eine unterschiedliche Sortierung der 

Die Transformation des um AP1 linearisierten Systems mit der MATLAB Funktion \lstinline[style=Matlab_colored]|canon| ergibt folgende Systemmmatritzen:
\begin{align*}
\textbf{A}_{\textrm{D}} &= \begin{bmatrix} 0& 16,0744 & 0& 0\\ -16,0744& 0 & 0 & 0\\ 0&0&&5,9929\\0&0&-5,9929&0\end{bmatrix}&
\textbf{B}_{\textrm{D}} &= \begin{bmatrix} 5,3736\\0\\0\\-2,1428\end{bmatrix}\\
\textbf{C}_{\textrm{D}} &= \begin{bmatrix} 0&-1,3739&-1,8832&0\\0&2,8813&-2,6939&0\end{bmatrix}&
\mathbf{D}_{\mathrm{D}} &= \begin{bmatrix} 0\\0\end{bmatrix}
\end{align*}

\section{Untersuchung von Steuerbarkeit und Beobachtbarkeit}
\lstinputlisting[style=Matlab_colored,caption={Code der Steuerbarkeitsfunktion nach Kalman}, label={Code_ctrbKalman}]{Codes/checkCtrbKalman.m}
\lstinputlisting[style=Matlab_colored,caption={Code der Beobachtbarbarkeitsfunktion nach Kalman}, label={Code_obsvKalman}]{Codes/checkObsvKalman.m}

Die MATLAB Funktionen \lstinline[style=Matlab_colored]|ctrb| und \lstinline[style=Matlab_colored]|obsv| ergeben die selben Steuerbar- beziehungsweise Beobachtbarkeitsmatritzen
\begin{align*}
\mathbf{S}_\mathrm{S\,AP1} &= 10000 \cdot \begin{bmatrix}0&0,0143&0&-3,1532\\0,0143&0&-3,1532&0\\0&-0,0214&0&6,3064\\-0,0214&0&6,3064&0\end{bmatrix}\\
\mathbf{S}_\mathrm{B\,AP1} &= \begin{bmatrix}1&0&0&0\\0&0&1&0\\0&1&0&0\\0&0&0&1\\-126,1286&0&63,0643&0\\189,1929&0&-168,1714&0\\0&-126,1286&0&63,0643\\0&189,1929&0&-168,1714\end{bmatrix}
\end{align*}
wie die selbst geschriebenen Funktionen \lstinline[style=Matlab_colored]|checkCtrbKalman| in \autoref{Code_ctrbKalman} und \lstinline[style=Matlab_colored]|checkObsvKalman| in \autoref{Code_obsvKalman}. Dies gilt f�r alle Arbeitspunkte
\lstinputlisting[style=Matlab_colored,caption={Code der Steuerbarkeitsfunktion nach Gilbert}, label={Code_ctrbGilbert}]{Codes/checkCtrbGilbert.m}
\lstinputlisting[style=Matlab_colored,caption={Code der Beobachtbarbarkeitsfunktion nach Gilbert}, label={Code_obsvGilbert}]{Codes/checkObsvGilbert.m}

\lstinputlisting[style=Matlab_colored,caption={Code der Steuerbarkeitsfunktion nach Hautus}, label={Code_ctrbHautus}]{Codes/checkCtrbHautus.m}
\lstinputlisting[style=Matlab_colored,caption={Code der Beobachtbarbarkeitsfunktion nach Hautus}, label={Code_obsvHautus}]{Codes/checkObsvHautus.m}
\end{document}
