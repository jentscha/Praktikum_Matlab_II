% =================================================================================
% Hier ausw�hlen, ob TUD-Design oder nicht
% =================================================================================
\newif\ifTUDdesign
\TUDdesigntrue					% TUD-Design
%\TUDdesignfalse				% F�r Rechner ohne installierte TUDdesign-Pakete
% =================================================================================


% =================================================================================
% Hier Daten f�r studentische Arbeit eingeben
% =================================================================================
\newcommand{\SADATyp}{Praktikumsbericht}
\newcommand{\SADATitel}{Versuch II: Linearisierung, Steuerbarkeit und Beobachtbarkeit}
\newcommand{\SADAStadt}{Darmstadt}
\newcommand{\SADAAutor}{Andreas Jentsch, Ali Kerem Sacakli}
\newcommand{\SADABetreuer}{}
\newcommand{\SADABetreuerII}{}
\newcommand{\SADABetreuerIII}{}
\newcommand{\SADABegin}{}
\newcommand{\SADAAbgabe}{Praktikum Matlab/Simulink II}
\newcommand{\SADASeminar}{}
% =================================================================================


% =================================================================================
% Auswahl des IAT-Fachgebiets (rtm / rtp)
% =================================================================================
\newif\ifrtm
\rtmtrue	% rtm
%\rtmfalse	% rtp
% =================================================================================


% =================================================================================
% Erkl�rung, dass die Arbeit ohne Hilfe Dritter etc. erstellt wurde
% =================================================================================
\def\SADAVarianteErklaerung{ETIT}		% FB 18, Elektrotechnik
%\def\SADAVarianteErklaerung{MBDA}		% FB 16, Maschinenbau, Diplomarbeit
%\def\SADAVarianteErklaerung{MBSA}		% FB 16, Maschinenbau, Studienarbeit
% =================================================================================


% =================================================================================
% Ausnahmen von der automatischen Silbentrennung
% =================================================================================
\hyphenation{Aktu-ali-sie-rung Screen-shots Pa-rallel-ro-bo-ter Zu-stands-raum-mo-del-le nach-voll-zieh-bar Pro-jekt-se-mi-nar}
% =================================================================================


% =================================================================================
% Hier NIICHTS �ndern!
% =================================================================================
\ifTUDdesign
	\documentclass[11pt, twoside, colorback, accentcolor=tud2c, nopartpage, bigchapter, fleqn, ngerman, longdoc]{tudreport}
\else
	\documentclass[11pt, a4paper, twoside, fleqn, ngerman]{scrreprt}
  % F�r Entwurf auf Rechnern ohne installierte TUDdesign-Pakete	
	\usepackage{exscale}	% Korrektur math-Zeichen
	\usepackage{eurosym}
\fi
%eps-plot von Matlab in pdf umwandeln, um diese einbinden zu k�nnen
\usepackage{epstopdf}
%Paket listings f�r das Einbinden von Quelltext
\usepackage{listings}
%Stil Matlab_colored definieren
\lstdefinestyle{Matlab_colored}
{
	language = Matlab,
	tabsize = 4,
	framesep = 3mm,
	frame = tb,
	classoffset = 0,
	basicstyle = \ttfamily,
	keywordstyle = \bfseries\color[rgb]{0,0,1},
	commentstyle = \itshape\color[rgb]{0.133,0.545,0.133},
	stringstyle = \color[rgb]{0.672,0.126,0.941},
	extendedchars = true,
	breaklines = true,
	prebreak = \textrightarrow,
	postbreak = \textleftarrow,
	numbers = left,
	numberstyle = \tiny,
	stepnumber = 5
}


\input{common/includes.tex}				% verwendete Pakete einbinden
\input{common/setup.tex}					% LaTeX-Einstellungen
\input{common/commonmacros.tex}		% oft verwendete Befehle
% =================================================================================


% =================================================================================
% Hier beginnt das eigentliche Dokument
% =================================================================================

\begin{document}
%\input{common/preface.tex} % Titelseite, Aufgabenstellung, Erkl�rung, Abstract, Inhaltsverzeichnis, etc.
\maketitle

\setcounter{chapter}{2}

%Warum geht title/section nicht?
\section{Linearisierung}
Im folgenden Abschnitt wird die Funktion zur Linearisierung des Doppelpendel-Systems um einen Arbeitspunkt, sowie ihre R�ckgabewerte an bestimmten Arbeitspunkten dokumentiert. Die Implementierung der Funktion ist in \autoref{Code_Linearisierung} aufgef�hrt.

Bevor das System linearisiert wird sind zwei Fragen zu kl�ren:
\begin{enumerate}
	\item Welche Arbeitspunkte sind sinnvoll?
	\item Was bedeutet es physikalisch, wenn $M_{\textrm{AP}}$ ungleich null ist?
\end{enumerate}
Die Antworten lauten wie folgt:
\begin{enumerate}
	\item Es ist nur sinnvoll das System in Arbeitspunkten zu linearisieren, in denen es sowohl vollst�ndig beobachtbar, als auch steuerbar ist.
	\item Bei der Gr��e $M_{\textrm{AP}}$ handelt es sich um den statischen Wert der Stellgr��e $M$ im Arbeitspunkt. Ist diese ungleich null muss der Motor das Moment $M_{\textrm{}}$ 
\end{enumerate}

In dieser Aufgabe soll eine Funktion \verb|[A,B,C,D]=linearisierung(f,h,AP)| implementiert werden. Unter Vorgabe der Funktionen \verb|f, h| und des Arbeitspunktes \verb|AP| sollen die Matrizen der linearisierten Gleichungen in Zustandsraumdarstellung ausgegeben werden.

\lstinputlisting[style=Matlab_colored,caption={Das vorgegebene nichtlineare Modell}, label={Code_Linearisierung}]{Codes/nonlinear_model.m}
\lstinputlisting[style=Matlab_colored,caption={Code der Linearisierungsfunktion}, label={Code_Linearisierung}]{Codes/linearisierung.m}

Die Linearisierung um die Arbeitspunkte
\begin{align*}
	\textbf{x}_{\textrm{AP}_1} &= \begin{bmatrix} 0 & 0 & 0 & 0\end{bmatrix}\\
	\textbf{x}_{\textrm{AP}_2} &= \begin{bmatrix} \pi & 0 & \pi & 0\end{bmatrix}\\
	\textbf{x}_{\textrm{AP}_3} &= \begin{bmatrix} \pi/2 & 0 & \pi & 0 \end{bmatrix}
\end{align*}
ergibt f�r die allgemeine Zustandsraumdarstellung:
\begin{align*}
	\textbf{\.{x}} &= \textbf{Ax + Bu}\\
	\textbf{y} &= \textbf{Cx + Du}
\end{align*}
die folgenden Systemmatrizen:


\begin{itemize}
	\item $\vec{x}_{AP1}$=$\begin{pmatrix}
	0&0&0&0
	\end{pmatrix}$
	\begin{itemize}
		\item $\bf{A_{AP1}}= \begin{pmatrix}
		0&1&0&0\\
		-126,1286&0&63,0643&0\\
		0&0&0&1\\
		189,1929&0&-168,1714&0
		\end{pmatrix}$
		
		\item $\bf{B_{AP1}}= \begin{pmatrix}
		0\\
		142,8571\\
		0\\
		-214,2857
		\end{pmatrix}$
		
		\item $\bf{C_{AP1}}= \begin{pmatrix}
		1&0&0&0\\
		0&0&1&0
		\end{pmatrix}$
		
		\item $\bf{D_{AP1}}= \begin{pmatrix}
		0\\
		0
		\end{pmatrix}$
		
	\end{itemize}
	
	
	\item $\vec{x}_{AP2}$=$\begin{pmatrix}
	\pi&0&\pi&0
	\end{pmatrix}$
	
	\begin{itemize}
		\item $\bf{A_{AP2}}= \begin{pmatrix}
		0&1&0&0\\
		126.1286&0&-63,0643&0\\
		0&0&0&1\\
		-189,1929&0&168,1714&0
		\end{pmatrix}$
		
		\item $\bf{B_{AP2}}= \begin{pmatrix}
		0\\
		142,8571\\
		0\\
		-214,2857
		\end{pmatrix}$
		
		\item $\bf{C_{AP2}}= \begin{pmatrix}
		1&0&0&0\\
		0&0&1&0
		\end{pmatrix}$
		
		\item $\bf{D_{AP2}}= \begin{pmatrix}
		0\\
		0
		\end{pmatrix}$
		
	\end{itemize}
	
	
	
	\item $\vec{x}_{AP3}$=$\begin{pmatrix}
	\pi/2&0&\pi&0
	\end{pmatrix}$
	
	\begin{itemize}
		\item $\bf{A_{AP3}}= \begin{pmatrix}
		0&1&0&0\\
		0&0&0&0\\
		0&0&0&1\\
		0&0&73,575&0
		\end{pmatrix}$
		
		\item $\bf{B_{AP3}}= \begin{pmatrix}
		0\\
		62,5\\
		0\\
		0
		\end{pmatrix}$
		
		\item $\bf{C_{AP3}}= \begin{pmatrix}
		1&0&0&0\\
		0&0&1&0
		\end{pmatrix}$
		
		\item $\bf{D_{AP3}}= \begin{pmatrix}
		0\\
		0
		\end{pmatrix}$
		
	\end{itemize}
	
	
\end{itemize}


\paragraph{Was bedeutet es physikalisch, wenn $M_{AP}$ ungleich null ist?}
 (TEXT)


\section{Vergleich der linearisierten Modelle}
Die Eigenwerte der Zustandsraummodelle um die drei Arbeitspunkte lauten wie folgt:

	%align hatte hier einen Fehler verursacht
	$\vec{x}_{AP1}$=$\begin{pmatrix}
	0&0&0&0 \end{pmatrix}$:
	 \newline
	$\bf{\lambda_1}=\begin{pmatrix}
	16,0744i\\-16,0744i\\5,9929i\\-5,9929i
	\end{pmatrix}$ \newline
	
	$\vec{x}_{AP2}$=$\begin{pmatrix}
	\pi&0&\pi&0 \end{pmatrix}$: \newline
	$\bf{\lambda_2}=\begin{pmatrix}
	-16,0744\\16,0744\\5,9929\\-5,9929
	\end{pmatrix}$
	
	
	$\vec{x}_{AP_3}$=$\begin{pmatrix}\pi/2&0&\pi&0 \end{pmatrix}$: \newline
	$\bf{\lambda_3}=\begin{pmatrix}
	-8,5776\\8,5776\\0\\0
	\end{pmatrix}$	\\
	
	\paragraph{Welche Unterschiede zwischen den Eigenwerte der Zustandsraummodelle liegen vor und worauf sind diese zur�ckzuf�hren? K�nnen Sie sich vorstellen, was sich �ndern w�rde, wenn die Reibung ber�cksichtigt w�re?}
	(TEXT)

\section{Normalformen des Zustandsraummodelles}
Es soll eine Funktion \verb| [AD,BD,CD,DD]=diagonalForm(A,B,C,D)| implementiert werden, die ein gegebenes System in Diagonalform transformiert.
\lstinputlisting[style=Matlab_colored,caption={Code der Diagonalisierungsfunktion}, label={Code_diagonalForm}]{Codes/diagonalForm.m}

Zus�tzlich soll das System im zweiten Arbeitspunkt anhand dieser Funktion und mit der Funktion \verb|canon| auf Diagonalform transformiert werden.

Die Ergebnisse der Funktion \verb|diagonalForm| sind hier wie folgt:

\begin{itemize}
	\item $\bf{A2D}= \begin{pmatrix}
	-16,0744&0&0&0\\
	0&16,0744&0&0\\
	0&0&-5,9929&0\\
	0&0&0&5,9929
	\end{pmatrix}$
	
	\item $\bf{B_{AP3}}= \begin{pmatrix}
	138,1293\\
	138,1293\\
	-21,3960\\
	21,3960
	\end{pmatrix}$
	
	\item $\bf{C_{AP3}}= \begin{pmatrix}
	-0,0267&0,0267&0,0943&0,0943\\
	0,0560&-0,0560&0,1349&0,1349
	\end{pmatrix}$
	
	\item $\bf{D_{AP3}}= \begin{pmatrix}
	0\\
	0
	\end{pmatrix}$
\end{itemize}	
	
Die Diagonalisierung mittels der Funktion \verb|canon| f�hrt zu folgendem Ergebnis:

\begin{itemize}	
	\item $\bf{A2D}= \begin{pmatrix}
	-16,07&0&0&0\\
	0&-5,993&0&0\\
	0&0&16,07&0\\
	0&0&0&5,993
	\end{pmatrix}$
	
	\item $\bf{B2D}= \begin{pmatrix}
	2,535\\
	-0,7024\\
	2,535\\
	0,7024
	\end{pmatrix}$
	
	\item $\bf{C2D}= \begin{pmatrix}
	-1,456&2,873&1,456&2,873\\
	3,053&4,109&-3,053&4,109
	\end{pmatrix}$
	
	\item $\bf{D2D}= \begin{pmatrix}
	0\\
	0
	\end{pmatrix}$	
	
\end{itemize}

\paragraph{Welche Unterschiede sind zu erkennen? Wie lassen sie sich erkl�ren?}
(TEXT)

Als n�chstes soll das System im ersten Arbeitspunkt auf Modalform transformiert werden. Das in Modalform transormierte System im ersten Arbeitspunkt sieht wie folgt aus:

%gaanz kleine Zahlen gebe ich im Folgenden als Null an
\begin{itemize}	
	\item $\bf{A2D}= \begin{pmatrix}
	0&16,07&0&0\\
	-16,07&0&0&0\\
	0&0&0&5,993\\
	0&0&-5,993&0
	\end{pmatrix}$
	
	\item $\bf{B2D}= \begin{pmatrix}
	5,374\\
	0\\
	0\\
	-2,143
	\end{pmatrix}$
	
	\item $\bf{C2D}= \begin{pmatrix}
	0&-1,374&-1,883&0\\
	0&2,881&-2,694&0
	\end{pmatrix}$
	
	\item $\bf{D2D}= \begin{pmatrix}
	0\\
	0
	\end{pmatrix}$	
	
\end{itemize}




\section{Untersuchung von Steuerbarkeit und Beobachtbarkeit}
Hier sollen die im Skript vorgestellten �berpr�fungsverfahren unter Verwendung der im Rahmen des Versuchs erstellten Funktionen implementiert werden. Die Funktionen f�r die �berpr�fungsverfahren sollen m�glichst allgemein verwendbar sein.

Die �berpr�fungsverfahren nach Kalman:
\lstinputlisting[style=Matlab_colored,caption={Code der Steuerbarkeitsfunktion nach Kalman}, label={Code_diagonalForm}]{Codes/checkCtrbKalman.m}
\lstinputlisting[style=Matlab_colored,caption={Code der Beobachtbarbarkeitsfunktion nach Kalman}, label={Code_diagonalForm}]{Codes/checkObsvKalman.m}

Vergleich mit ctrb und obsv

\lstinputlisting[style=Matlab_colored,caption={Code der Steuerbarkeitsfunktion nach Gilbert}, label={Code_diagonalForm}]{Codes/checkCtrbGilbert.m}
\lstinputlisting[style=Matlab_colored,caption={Code der Beobachtbarbarkeitsfunktion nach Gilbert}, label={Code_diagonalForm}]{Codes/checkObsvGilbert.m}

\lstinputlisting[style=Matlab_colored,caption={Code der Steuerbarkeitsfunktion nach Hautus}, label={Code_diagonalForm}]{Codes/checkCtrbKalman.m}
\lstinputlisting[style=Matlab_colored,caption={Code der Beobachtbarbarkeitsfunktion nach Hautus}, label={Code_diagonalForm}]{Codes/checkObsvKalman.m}
\end{document}
