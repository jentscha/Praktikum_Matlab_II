%
\chapter{Verzeichnisstruktur und vordefinierte Befehle der IAT-Vorlage}\label{cha:Verzeichnisstruktur}


Es handelt sich bei diesem \LaTeX-Dokument um ein f�r studentische Arbeiten am Institut f�r Automatisierungstechnik vorbereitetes Dokument auf Basis der Klasse \verb|tudreport|. Es ist keine neue, abgeleitete Klasse definiert!

Die Klasse \verb|tudreport| ist aus der Standard-Klasse \verb|scrreprt| abgeleitet
und stellt nur wenige neue Befehle zur Verf�gung; weitere Funktionen k�nnen bei
Bedarf durch Zusatzpakete eingebunden oder selbst definiert werden. Die Klasse
ist daher auch so aufgebaut, dass sie mit m�glichst vielen Paketen zusammen
arbeitet. Im Wesentlichen wird das Layout angepasst, wie es in \cite{Richtlinien}
festgelegt ist und sich f�r solche Arbeiten bew�hrt hat, \zB:
\begin{itemize}
\item Es wird doppelseitig auf DIN-A4-Papier geschrieben. In die zu erstellende
  PDF-Version werden Bookmarks und Hyperlinks (nicht farbig!) integriert.
\item Der Abstand der Zeilen betr�gt das 1,25-fache des Standard-Abstands von
  \LaTeX. Da technische Arbeiten viele Formeln und Bilder enthalten, werden
  Abs�tze durch einen zus�tzlichen vertikalen Zwischenraum statt durch einen
  Einzug getrennt.
\item Kapitel beginnen immer auf einer neuen Seite.
\item Die Titelseite hat ein festes Layout mit dem Logo der TU~Darmstadt.
\end{itemize}
%
%
\section{Verzeichnisse}
\begin{itemize}
	\item \verb|bib|\\Hier wird standardm��ig die Datei \verb|literature.bib| mit den Bibtex-Eintr�gen erwartet.
	\item \verb|common|\\Allgemeinere Dateien, in die Teile der Definitionen ausgelagert sind, damit die Hauptdatei nicht �berfrachtet wird.
	\item \verb|images|\\Vorgesehen f�r Bilder
	\item \verb|inc|\\Vorgesehen f�r tex-Dateien mit eigentlichem Inhalt
\end{itemize}


\section{Inhalt Verzeichnis \texttt{common}}
Damit das Hauptdokument nicht �berfrachtet wird, sind die folgenden l�ngeren "`Abschnitte"' in die angegebenen Dateien im Unterverzeichnis \verb|common| ausgelagert:
\begin{itemize}
    \item \verb|commonmacros.tex| \\
		Enth�lt eine Reihe n�tzlicher selbst definierter Befehle, Mathe- und Beispielumgebungen sowie
        die Formate \verb|Matlab_colored_smallfont| und \verb|Matlab_colored| zur formatierten Darstellung von Matlab-Code (siehe \charef{cha:commonmacros} f�r eine �bersicht).
%
	\item \verb|includes.tex| \\
		Beinhaltet alle \verb|\usepackage|-Befehle
	\item \verb|preface.tex| \\
		Generiert die ersten Seiten der Arbeit (Aufgabenstellung, Erkl�rung, Inhaltsverzeichnis, \etc und nimmt weitere Einstellungen vor)
	\item \verb|SADA_Abstract.tex| \\
		Kurzfassung der Arbeit in deutscher und englischer Sprache.
  \item \verb|SADA_Aufgabenstellung.tex| \\
		Aufgabenstellung bei einer studentischen Arbeit. Achtung: f�r den FB16 muss f�r das offizielle Exemplar die im Original unterschriebene Aufgabenstellung an dieser Stelle mit gebunden werden.
	\item \verb|SADA_Erklaerung.tex| \\
		Erkl�rung zu studentischen Arbeiten. (Je nach Fachbereich und Art der Arbeit muss \ggf etwas angepasst werden.)
	\item \verb|setup.tex| \\
        Nimmt generelle Einstellungen vor. Diese sollten nur von kundigen Benutzern ge�ndert werden.
	\item Logos \\
        Au�erdem sind im Verzeichnis \texttt{common} noch die Logo-Grafiken im eps- und pdf-Format abgelegt.
\end{itemize}

\section{Angaben �ber die Arbeit}
Es sind im Hauptdokument die Befehle
\begin{verbatimtab}
	\newcommand{\SADATyp}{Diplomarbeit}
    \newcommand{\SADATitel}{Titel der Arbeit}
    \newcommand{\SADAStadt}{Darmstadt}
    \newcommand{\SADAAutor}{Martin Mustermann}
    \newcommand{\SADABetreuer}{Dipl.-Ing. Rudi Ratlos,}
    \newcommand{\SADABetreuerII}{Dipl.-Ing. Hans Hilflos}
    \newcommand{\SADABetreuerIII}{}
    \newcommand{\SADABegin}{01. Januar 2000}
    \newcommand{\SADAAbgabe}{01. Juli 2000}
    \newcommand{\SADASeminar}{01. August 2000}
\end{verbatimtab}
sinnvoll zu setzen.

Der Fachbereich 16 (Maschinenbau) schreibt je nachdem, ob es sich um eine Studien- oder Diplomarbeit handelt, unterschiedliche Angaben auf der "`Erkl�rungsseite"' vor. Es ist im Hauptdokument eine der drei Zeilen
\begin{verbatimtab}
	\def\SADAVarianteErklaerung{ETIT}
	%\def\SADAVarianteErklaerung{MBDA}
	%\def\SADAVarianteErklaerung{MBSA}
\end{verbatimtab}
einzukommentieren. (\verb|ETIT|: "`Normal"', \verb|MBDA|: Fachbereich 16, Diplomarbeit, \verb|MBSA|: Fachbereich 16, Studienarbeit.)

Je nach Fachgebiet ist der Schalter
\begin{verbatimtab}
    \rtmtrue	% rtm
    %\rtmfalse	% rtp
\end{verbatimtab}
einzukommentieren.

\section{"`Entwurfsmodus"'}
Damit man nicht gezwungen ist, das tuddesign-Paket und die TU-Schriftarten auf jedem Rechner zu installieren, um mit dem Dokument zu arbeiten, ist es vorgesehen, auch die Klasse \verb|scrreprt| behelfsweise als Basis zu verwenden. Dazu ist Folgendes zu beachten:

Je nachdem, welche Klasse als Basis ausgew�hlt ist, muss der Schalter
\begin{verbatimtab}
	\TUDdesigntrue    % TUD-Design
	%\TUDdesignfalse  % F�r Rechner ohne installierte TUDdesign-Pakete
\end{verbatimtab}
passend gesetzt werden. Dadurch wird erreicht, dass das Dokument auch mit \verb|scrreprt| fehlerfrei umgewandelt wird und die Seitenr�nder grob stimmen, so dass man einen Eindruck vom sp�teren Layout bekommt.



