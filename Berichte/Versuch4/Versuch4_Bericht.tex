% =================================================================================
% Hier ausw�hlen, ob TUD-Design oder nicht
% =================================================================================
\newif\ifTUDdesign
\TUDdesigntrue					% TUD-Design
%\TUDdesignfalse				% F�r Rechner ohne installierte TUDdesign-Pakete
% =================================================================================


% =================================================================================
% Hier Daten f�r studentische Arbeit eingeben
% =================================================================================
\newcommand{\SADATyp}{Praktikumsbericht}
\newcommand{\SADATitel}{Versuch IV: Beobachterentwurf - Benutzeroberfl�chen}
\newcommand{\SADAStadt}{Darmstadt}
\newcommand{\SADAAutor}{Andreas Jentsch, Ali Kerem Sacakli}
\newcommand{\SADABetreuer}{}
\newcommand{\SADABetreuerII}{}
\newcommand{\SADABetreuerIII}{}
\newcommand{\SADABegin}{}
\newcommand{\SADAAbgabe}{Praktikum Matlab/Simulink II \newline 27. Juni 2017}
\newcommand{\SADASeminar}{}
% =================================================================================


% =================================================================================
% Auswahl des IAT-Fachgebiets (rtm / rtp)
% =================================================================================
\newif\ifrtm
\rtmtrue	% rtm
%\rtmfalse	% rtp
% =================================================================================


% =================================================================================
% Erkl�rung, dass die Arbeit ohne Hilfe Dritter etc. erstellt wurde
% =================================================================================
\def\SADAVarianteErklaerung{ETIT}		% FB 18, Elektrotechnik
%\def\SADAVarianteErklaerung{MBDA}		% FB 16, Maschinenbau, Diplomarbeit
%\def\SADAVarianteErklaerung{MBSA}		% FB 16, Maschinenbau, Studienarbeit
% =================================================================================


% =================================================================================
% Ausnahmen von der automatischen Silbentrennung
% =================================================================================
\hyphenation{Aktu-ali-sie-rung Screen-shots Pa-rallel-ro-bo-ter Zu-stands-raum-mo-del-le nach-voll-zieh-bar Pro-jekt-se-mi-nar}
% =================================================================================


% =================================================================================
% Hier NIICHTS �ndern!
% =================================================================================
\ifTUDdesign
	\documentclass[11pt, twoside, colorback, accentcolor=tud2c, nopartpage, bigchapter, fleqn, ngerman, longdoc]{tudreport}
\else
	\documentclass[11pt, a4paper, twoside, fleqn, ngerman]{scrreprt}
  % F�r Entwurf auf Rechnern ohne installierte TUDdesign-Pakete	
	\usepackage{exscale}	% Korrektur math-Zeichen
	\usepackage{eurosym}
\fi
%To Do Package
\usepackage{todonotes}
%eps-plot von Matlab in pdf umwandeln, um diese einbinden zu k�nnen
\usepackage{epstopdf}
%Paket listings f�r das Einbinden von Quelltext
\usepackage{listings}
%Stil Matlab_colored definieren
\lstdefinestyle{Matlab_colored}
{
	language = Matlab,
	tabsize = 4,
	framesep = 3mm,
	frame = tb,
	classoffset = 0,
	basicstyle = \ttfamily,
	keywordstyle = \bfseries\color[rgb]{0,0,1},
	commentstyle = \itshape\color[rgb]{0.133,0.545,0.133},
	stringstyle = \color[rgb]{0.672,0.126,0.941},
	extendedchars = true,
	breaklines = true,
	prebreak = \textrightarrow,
	postbreak = \textleftarrow,
	numbers = left,
	numberstyle = \tiny,
	stepnumber = 5
}


\input{common/includes.tex}				% verwendete Pakete einbinden
\input{common/setup.tex}					% LaTeX-Einstellungen
\input{common/commonmacros.tex}		% oft verwendete Befehle
% =================================================================================


% =================================================================================
% Hier beginnt das eigentliche Dokument
% =================================================================================

\begin{document}
%\input{common/preface.tex} % Titelseite, Aufgabenstellung, Erkl�rung, Abstract, Inhaltsverzeichnis, etc.
\maketitle

\setcounter{chapter}{4}
\setcounter{section}{7}
\section{Verhalten des Regelkreises mit Beobachter}
%\todo[inline]{Aufgabe 1.4 - auch Verl�ufe/Screenshots einf�gen!?} ja

Nachfolgend sind Verl�ufe der theoretisch Messbaren Zustandsgr��en, sowie der Stellgr��e abgebildet, um den Einfluss von verschiedenen Beobachtereigenwerten darzustellen. Dabei sind die vom Beobachter gesch�tzten Zust�nden zusammen mit den realwerten abgebildet.

Die Eigenwerte des Geschlossenen Regelkreises liegen bei,
\begin{align*}
\lambda &= \begin{bmatrix}
-1,3152& -5,6606-3,0304i& -5,6606+3,0304i&-258,5879 
\end{bmatrix}
\end{align*}
Der Arbeitspunkt, sowie die Startwerte des Systems und des Beobachters liegen bei
\begin{align*}
\mathbf{x}_{\mathrm{AP}} &=\begin{bmatrix}\pi&0&\pi&0\end{bmatrix}\\
\mathbf{x}_0^{\mathrm{T}} &=\begin{bmatrix}3&0&\pi&0\end{bmatrix}\\
\mathbf{x}_{\mathrm{obs},0}^{\mathrm{T}} &=\begin{bmatrix}\pi&0&\pi&0\end{bmatrix}
\end{align*}

In \autoref{fig:beob1} bis \ref{fig:beob4} werden zunehmend schneller ausgelegte Beobachter verwendet. Beim langsamsten Beobachter in \autoref{fig:beob1} ist neben der langsamen Ausregelung des Sch�tzfehlers eine niedrige maximale Stellgr��e festzustellen, das System regelt den Winkel $\varphi_1$ infolgedessen sehr langsam aus. Hier liegen die dominanten Eigenwerte des Beobachters noch Rechts des dominanten Systemeigenwertes. Die �brigen Abbildungen zeigen den folgenden Trend: Je weiter links der Systemeigenwerte die  Beobachtereigenwerte liegen, desto schneller wird der Beobachtungsfehler ausgeregelt. Damit steigt auch die Maximale Stellgr��e und das System wird schneller in den Arbeitspunkt �berf�hrt.

Bei den Werten f�r die gemessenen Zust�nde $x_1$ und $x_3$, welche den Winkeln $\varphi_1$ beziehungsweise $\varphi_2$ entsprechen f�llt auf, dass kein Sch�tzfehler entsteht, wenn die Anfangswerte von Beobachter und System �bereinstimmen. Dies ist bei allen Verl�ufen f�r $\varphi_2$ zu sehen. Um den anf�nglichen Sch�tzfehler bei $\varphi_1$ zu minimieren, k�nnten die Anfagswerte des Beobachters an die des Systems angepasst werden, $\mathbf{x}_{\mathrm{obs},0} = \mathbf{x}_{0}$.


\begin{figure}[htbp]
	\centering
	\includegraphics[width = \textwidth]{Bilder/Aufgabe4_8_beob_1_1_5_5.eps}
	\caption{Systemverhalten mit Beobachtereigenwerten bei: [-1 -1 -5 -5]}
	\label{fig:beob1}
\end{figure}

\begin{figure}[htbp]
	\centering
	\includegraphics[width = \textwidth]{Bilder/Aufgabe4_8_beob_5_5_10_10.eps}
	\caption{Systemverhalten mit Beobachtereigenwerten bei: [-5 -5 -10 -10]}
	\label{fig:beob2}
\end{figure}

\begin{figure}[htbp]
	\centering
	\includegraphics[width = \textwidth]{Bilder/Aufgabe4_8_beob_10_10_20_20.eps}
	\caption{Systemverhalten mit Beobachtereigenwerten bei: [-10 -10 -20 -20]}
	\label{fig:beob3}
\end{figure}

\begin{figure}[htbp]
	\centering
	\includegraphics[width = \textwidth]{Bilder/Aufgabe4_8_beob_20_20_40_40.eps}
	\caption{Systemverhalten mit Beobachtereigenwerten bei: [-20 -20 -40 -40]}
	\label{fig:beob4}
\end{figure}
\newpage

\section{GUI Entwurf}

Unter Ausnutzung der schon erstellten Funktionen soll in diesem Versuch das Modell durch eine grafische Benutzeroberfl�che (\autoref{fig:gui})und einen Luenberger-Beobachter erweitert werden. Die wichtigsten Callback-Funktionen sind in \autoref{berechneK_Callback} bis \ref{stopAnimation_Callback} aufgef�hrt.

% Zur Protokollierung des Versuchs werden im Folgenden die Ergebnisse als Listings und Screenshots vorgestellt.

\begin{figure}[htbp]
	\centering
	\includegraphics[width = \textwidth]{Bilder/GUI.JPG}
	\caption{Finale grafische Benutzeroberfl�che}
	\label{fig:gui}
\end{figure}

%\todo[inline]{Codes besser kommentieren}
%\lstinputlisting[style=Matlab_colored,caption={Quellcode der relevanten Callback-Funktionen}, label={callback}]{Codes/simGUI.m}

\lstinputlisting[style=Matlab_colored,caption={Quellcode der Callback-Funktion zur Reglerberechnung}, label={berechneK_Callback}]{Codes/berechneK_Callback.m}

\lstinputlisting[style=Matlab_colored,caption={Quellcode Callback-Funktion zum Start der Simulation}, label={startSim_Callback}]{Codes/startSim_Callback.m}

\lstinputlisting[style=Matlab_colored,caption={Quellcode der Callback-Funktion zum Stoppen der Animation}, label={stopAnimation_Callback}]{Codes/stopAnimation_Callback.m}


\section{Beobachterentwurf}
Zur Berechnung der Beobachter-Matrix \verb|L| soll die Funktion \verb|berechneBeobachter| implementiert werden.     
\lstinputlisting[style=Matlab_colored,caption={Quellcode der Funktion berechneBeobachter}, label={berechneBeobachter}]{Codes/berechneBeobachter.m}

F�r die Erweiterung soll zudem die Funktion \verb|runPendel| erweitert werden:
\lstinputlisting[style=Matlab_colored,caption={Quellcode der Funktion runPendel}, label={runPendel}]{Codes/runPendel.m}

Das Zugrundeliegende Simulink Modell mit Beobachter ist in \autoref{fig:simObs} zu sehen.

\begin{figure}[htbp]
	\centering
	\includegraphics[width = \textwidth]{Bilder/simulink_V4.JPG}
	\caption{Simulink-Modell mit Luenberger-Beobachter}
	\label{fig:simObs}
\end{figure}

\end{document}
