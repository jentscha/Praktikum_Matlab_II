\section*{Aufgabenstellung}
F�r schriftliche Arbeiten (Pro-/Projektseminar, Studien-, Bachelor-, Master-, Diplomarbeiten, \etc) soll Studierenden ein \LaTeX-Dokument zur Verf�gung gestellt werden, das die Vorgaben aus den \emph{Richtlinien zur Anfertigung von Studien- und Diplomarbeiten}~\cite{Richtlinien} umsetzt. Die Dokumentation soll die Funktionen des Dokumentes beschreiben und Hinweise zu ihrer Anwendung geben.

Grundlage ist die \verb|tudreport|-Klasse. Die damit erstellten Arbeiten m�ssen sowohl zum Ausdrucken geeignet sein als auch f�r die Bildschirmdarstellung und die elektronische Archivierung als PDF-Datei.

\vspace{0.5cm}
\begin{tabular}{ll}
Beginn: & \SADABegin \\
Ende:   & \SADAAbgabe \\
Seminar:& \SADASeminar
\end{tabular}

\vspace{1cm}

\begin{tabular}{ll}
\rule{7cm}{0.4pt} \hspace{1cm} & \rule{7cm}{0.4pt} \\
\SADAProf & \SADABetreuer\\
 &\SADABetreuerII\\
 &\SADABetreuerIII
\end{tabular}

\vfill
{\renewcommand{\baselinestretch}{1} % f�r diesen Abschnitt einfacher Zeilenabstand
\normalsize % anwenden des Zeilenabstandes
\begin{minipage}{0.8\textwidth}
	Technische Universit�t Darmstadt\\
	\SADAinstitut\\[0.5cm]
%
	Landgraf-Georg-Stra�e 4\\
	64283 Darmstadt\\
	Telefon \SADAtel\\
	\SADAwebsite
\end{minipage}
\begin{minipage}{0.2\textwidth}
\flushright  % rechtsb�ndig
\ \\[2.7cm]
\SADAlogo\;
\end{minipage}}

