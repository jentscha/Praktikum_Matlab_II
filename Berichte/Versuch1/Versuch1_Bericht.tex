% =================================================================================
% Hier ausw�hlen, ob TUD-Design oder nicht
% =================================================================================
\newif\ifTUDdesign
\TUDdesigntrue					% TUD-Design
%\TUDdesignfalse				% F�r Rechner ohne installierte TUDdesign-Pakete
% =================================================================================


% =================================================================================
% Hier Daten f�r studentische Arbeit eingeben
% =================================================================================
\newcommand{\SADATyp}{Praktikumsbericht}
\newcommand{\SADATitel}{Versuch I}
\newcommand{\SADAStadt}{Darmstadt}
\newcommand{\SADAAutor}{Ali Kerem, Andreas Jentsch}
\newcommand{\SADABetreuer}{}
\newcommand{\SADABetreuerII}{}
\newcommand{\SADABetreuerIII}{}
\newcommand{\SADABegin}{}
\newcommand{\SADAAbgabe}{Praktikum Matlab/Simulink II}
\newcommand{\SADASeminar}{}
% =================================================================================


% =================================================================================
% Auswahl des IAT-Fachgebiets (rtm / rtp)
% =================================================================================
\newif\ifrtm
\rtmtrue	% rtm
%\rtmfalse	% rtp
% =================================================================================


% =================================================================================
% Erkl�rung, dass die Arbeit ohne Hilfe Dritter etc. erstellt wurde
% =================================================================================
\def\SADAVarianteErklaerung{ETIT}		% FB 18, Elektrotechnik
%\def\SADAVarianteErklaerung{MBDA}		% FB 16, Maschinenbau, Diplomarbeit
%\def\SADAVarianteErklaerung{MBSA}		% FB 16, Maschinenbau, Studienarbeit
% =================================================================================


% =================================================================================
% Ausnahmen von der automatischen Silbentrennung
% =================================================================================
\hyphenation{Aktu-ali-sie-rung Screen-shots Pa-rallel-ro-bo-ter Zu-stands-raum-mo-del-le nach-voll-zieh-bar Pro-jekt-se-mi-nar}
% =================================================================================


% =================================================================================
% Hier NIICHTS �ndern!
% =================================================================================
\ifTUDdesign
	\documentclass[11pt, twoside, colorback, accentcolor=tud2c, nopartpage, bigchapter, fleqn, ngerman, longdoc]{tudreport}
\else
	\documentclass[11pt, a4paper, twoside, fleqn, ngerman]{scrreprt}
  % F�r Entwurf auf Rechnern ohne installierte TUDdesign-Pakete	
	\usepackage{exscale}	% Korrektur math-Zeichen
	\usepackage{eurosym}
\fi
\input{common/includes.tex}				% verwendete Pakete einbinden
\input{common/setup.tex}					% LaTeX-Einstellungen
\input{common/commonmacros.tex}		% oft verwendete Befehle
% =================================================================================


% =================================================================================
% Hier beginnt das eigentliche Dokument
% =================================================================================
\begin{document}
%\input{common/preface.tex} % Titelseite, Aufgabenstellung, Erkl�rung, Abstract, Inhaltsverzeichnis, etc.
\maketitle



% =================================================================================
% Anhang
% =================================================================================
%\appendix % Damit wird der Anhang begonnen. Die Kapitel werden ab jetzt mit Buchstaben nummeriert



% =================================================================================


%% =================================================================================
%% Abbildungsverzeichnis
%% =================================================================================
%\cleardoublepage
%\phantomsection					% F�r Aufnahme ins Inhaltsverzeichnis
%\addcontentsline{toc}{chapter}{\listfigurename}	% In Inhaltsverzeichnis von
%												% Dokument und pdf aufnehmen
%\listoffigures
%% =================================================================================
%
%% =================================================================================
%% Tabellenverzeichnis
%% =================================================================================
%\cleardoublepage
%\phantomsection					% F�r Aufnahme ins Inhaltsverzeichnis
%\addcontentsline{toc}{chapter}{\listtablename}	% In Inhaltsverzeichnis von
%												% Dokument und pdf aufnehmen
%\listoftables
%% =================================================================================

% =================================================================================
% Literaturverzeichnis
% =================================================================================
\cleardoublepage
\phantomsection					% F�r Aufnahme ins Inhaltsverzeichnis
\addcontentsline{toc}{chapter}{\bibname}	% In Inhaltsverzeichnis von
											% Dokument und pdf aufnehmen
%\bibliographystyle{gerabbrv}	% Verweise nummeriert in eckigen Klammern, alphabetisch sortiert
\bibliographystyle{gerunsrt}	% Verweise nummeriert in eckigen Klammern, nach Erscheinung sortiert

% =================================================================================

\end{document}
