\chapter{Einf�hrung}\label{cha:Intro}
%
Beim Anfertigen einer Abschlussarbeit steht man als Studierender meist zum
ersten Mal vor dem Problem, einen l�ngeren wissenschaftlichen Text mit Bildern,
Gleichungen und Referenzen schreiben zu m�ssen. Daf�r bietet sich das
Textsatz-System \LaTeX\ an, zu dessen Vorteilen die weitgehende Trennung von
Inhalt und Layout geh�ren. Damit sich Studierende mehr mit dem \emph{Inhalt} der
Arbeit besch�ftigen k�nnen, stellt das IAT ein \LaTeX-Dokument zur Verf�gung, in
dem das \emph{Layout} der \verb|tudreport|-Klasse f�r das IAT angepasst wurde. Auch der vorliegende Text ist beispielhaft mit dieser Klasse geschrieben. \LaTeX-Grundlagen findet man \zB\ in \cite{Kopka}
oder \cite{Schmidt}. Allgemeines �ber
wissenschaftliche Arbeiten findet man in \cite{Friedrich}; als Einstieg in
die Typografie ist \cite{Willberg} sehr zu empfehlen.

Die Arbeit kann in Deutsch oder wahlweise auch in Englisch verfasst werden.

In der vorliegenden Anleitung werden in Kapitel \ref{cha:Hinweise} zuerst allgemeine Hinweise und Tipps zur Durchf�hrung einer wissenschaftlichen Arbeit gegeben. Im Anschlu� daran gibt Kapitel~\ref{cha:Hinweise-Latex} allgemeine Hinweise zum Erstellen der schriftlichen Arbeit. Es werden die \LaTeX-spezifischen Befehle vorgestellt, mit denen die Hinweise umgesetzt werden k�nnen. Die Hinweise in Kapitel~\ref{cha:Hinweise-Latex} sind auch f�r Studierende relevant, die sich gegen die Erstellung der Arbeit mit \LaTeX\ entscheiden. Da das Layout der Arbeit in diesem Fall zus�tzlich selbst erstellt werden muss, dient die vorliegende Anleitung als Vorlage. In Kapitel \ref{cha:Verzeichnisstruktur} wird schlie�lich beschrieben, wie mit der \LaTeX-Vorlage gearbeitet werden sollte. 

Im Anhang findet sich eine Checkliste die vor Abgabe der Arbeit unbedingt abgearbeitet werden sollte, um zu pr�fen, ob alle Vorgaben und Hinweise beachtet wurden.

%Im Anschluss daran finden sich in zwei Kapiteln Hinweise zur Erstellung der schriftlichen Arbeit. Kapitel~\ref{cha:Hinweise-Allgemein} gibt allgemeine Hinweise, wohingegen in Kapitel~\ref{cha:Hinweise-Latex} zus�tzlich \LaTeX-spezifische Befehle vorgestellt werden. Je nach verwendetem Programm, kann das eine oder andere Kapitel �bersprungen werden.

